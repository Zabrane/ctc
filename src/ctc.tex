% ctc.tex
\documentclass[10pt,a4paper]{article}
%
\begin{document}
%
%\title{Call--threaded virtual machine interpreters}
\title{Call Threaded Code}
\author{Mikael Pettersson\\
	\texttt{mikael.pettersson@acm.org} \thanks{Work done circa 2004 at the
		Department of Information Technology,
		Division of Computing Science,
		Uppsala University, Sweden}}
\date{	DRAFT}
\maketitle{}
%
\begin{abstract}
Virtual machine interpreters using the direct threading
technique execute large numbers of indirect jumps.
These indirect jumps tend to experience high branch prediction
miss rates on contemporary processors, resulting in very high
runtime overheads.
%
This paper describes \emph{call threaded code}, in which
the handlers for virtual machine instructions are
dispatched using matching pairs of hardware call and
return instructions. The primary benefit of this
technique is that it allows modern processors with
\emph{return address stack} branch predictors to execute
the dispatch code without any branch prediction misses.
A secondary benefit is that the interpreter can dispense
with its \emph{virtual program counter}.
%
To evaluate the technique a stack--oriented virtual machine was
implemented with both direct threading and call threading
on several different processor architectures.
On average, the call threaded interpreters were about twice
as fast as the direct threaded interpreters.
\end{abstract}
%
\section{Introduction}
Direct threading is a common implementation technique
used by efficient interpreters for virtual machines.
It represents each instance of a virtual machine instruction
by the address of the \emph{handler} (machine code block) that
interprets instructions of that type. 
Each handler ends with a sequence to \emph{dispatch}
the next instruction: loading the next instruction via
the virtual machine program counter, incrementing the
virtual machine program counter, and jumping to the address just loaded.
Compared to simple \emph{token threaded} (bytecoded) interpreters,
direct threading improves performance by avoiding the need to
translate instruction tokens (opcodes) to handler addresses
in the dispatch path, but the indirect jumps tend to have
unpredictable targets, causing frequent branch prediction
misses and high runtime overheads on modern processors.

Common approaches to reduce this overhead include
superinstructions, partial native code compilation,
duplicating the interpreter's handlers, and switching
from stack to register oriented virtual machines.
While they often lead to improved performance, they can
also require significant work to implement, and some of
them are also non-portable leading to higher maintenance
costs.

This paper describes \emph{call threading} and how it can
be used to implement efficient virtual machine interpreters.
Simpler forms of call threading were used in the 1970s,
but they quickly fell into disuse, largely because the
processors at that time could execute a load and indirect jump
faster than a function call.
Modern processors have changed the relative costs of these
operations. In particular, the return address stack branch
predictor can allow a pair of a call and a return instruction
to execute much faster than an indirect jump instruction that
experiences a branch prediction miss.

This paper describes in detail the implementation of call
threading for a simple stack--oriented virtual machine,
including the implementation of virtual machine instructions with
control flow effects or immediate operands. Implementation details
are given for the x86, AMD64, PowerPC, SPARC, and ARM architectures.
The amount of machine--dependent code in this implementation is
surprisingly small -- between 25 and 50 lines per processor architecture.

To evaluate the method measurements were taken using hardware performance
counters on the supported architectures. The results indicate that
call threading results in significant reduction of branch
prediction misses, some reduction in the number of instructions
executed, and significant reduction in clock cycles.
On average, the call threaded interpreters were about twice
as fast as the direct threaded interpreters.
%
\section{The TAK machine}
The TAK machine is a simple stack--oriented
virtual machine, with instructions for stack accesses,
arithmetic, unconditional and conditional jumps,
and function calls and returns. It has two registers:
a program counter pointing to an array of bytes,
and a stack pointer pointing to an array of integers.
The definition
of the machine's instructions is shown in Figure~\ref{fig:the-tak-machine}.
The TAK machine was designed to support running a compiled
version of the Takeuchi function. The original functional
source code of the Takeuchi function is shown in Figure~\ref{fig:tak-erl};
the corresponding TAK virtual machine code in token threaded form is shown in
Figure~\ref{fig:the-tak-program}.
%
\begin{figure}[htb]
\begin{center}
\begin{tabular}{@{}ll}
GET & \verb@x = sp[*pc++]; *--sp = x; NEXT;@ \\
SET & \verb@x = *sp++; sp[*pc++] = x; NEXT;@ \\
SUB1 & \verb@sp[0] -= 1; NEXT;@ \\
RET & \verb@x = *sp++; n = *pc; pc = *sp++; sp += n; *--sp = x; NEXT;@ \\
CALL & \verb@*--sp = pc+1; pc += *pc; NEXT;@ \\
JMP & \verb@pc += *pc; NEXT;@ \\
JGT & \verb@x = *sp++; y = *sp++; pc += (x > y) ? *pc : 1; NEXT;@ \\
PUSH & \verb@*--sp = *pc++; NEXT;@ \\
POP & \verb@++sp; NEXT;@ \\
HALT & execution halts
\end{tabular}
\end{center}
\caption{The TAK machine instructions}
\label{fig:the-tak-machine}
\end{figure}
%
\begin{figure}[htb]
\begin{verbatim}
tak(X, Y, Z) when X =< Y ->
   Z;
tak(X, Y, Z) ->
   A1 = tak(X-1, Y, Z),
   A2 = tak(Y-1, Z, X),
   A3 = tak(Z-1, X, Y),
   tak(A1, A2, A3).

tak() -> 
   tak(18, 12, 6).
\end{verbatim}
\caption{The functional code for TAK}
\label{fig:tak-erl}
\end{figure}
%
\begin{figure}[p]
\begin{verbatim}
#define NITER 255
#define Ltak3 1
/*  0 *//* tak3: */                     /* z y x ra */
/*  0 */        OP_DEBUG,               /* z y x ra */
/*  1 */        OP_GET, 2,              /* z y x ra y */
/*  3 */        OP_GET, 2,              /* z y x ra y x */
/*  5 */        OP_JGT, 11-6,           /* z y x ra */
/*  7 */        OP_GET, 3,              /* z y x ra z */
/*  9 */        OP_RET, 3,              /* z */
/* 11 */        OP_GET, 3,              /* z y x ra z */
/* 13 */        OP_GET, 3,              /* z y x ra z y */
/* 15 */        OP_GET, 3,              /* z y x ra z y x */
/* 17 */        OP_SUB1,                /* z y x ra z y x' */
/* 18 */        OP_CALL, Ltak3-19,      /* z y x ra a */
/* 20 */        OP_GET, 2,              /* z y x ra a x */
/* 22 */        OP_GET, 5,              /* z y x ra a x z */
/* 24 */        OP_GET, 5,              /* z y x ra a x z y */
/* 26 */        OP_SUB1,                /* z y x ra a x z y' */
/* 27 */        OP_CALL, Ltak3-28,      /* z y x ra a b */
/* 29 */        OP_GET, 4,              /* z y x ra a b y */
/* 31 */        OP_GET, 4,              /* z y x ra a b y x */
/* 33 */        OP_GET, 7,              /* z y x ra a b y x z */
/* 35 */        OP_SUB1,                /* z y x ra a b y x z' */
/* 36 */        OP_CALL, Ltak3-37,      /* z y x ra a b c */
/* 38 */        OP_SET, 5,              /* c y x ra a b */
/* 40 */        OP_SET, 3,              /* c b x ra a */
/* 42 */        OP_SET, 1,              /* c b a ra */
/* 44 */        OP_JMP, Ltak3-45,       /* c b a ra */
/* 46 *//* tak0: */                     /* */
/* 46 */        OP_PUSH, 6,             /* ra 6 */
/* 48 */        OP_PUSH, 12,            /* ra 6 12 */
/* 50 */        OP_PUSH, 18,            /* ra 6 12 18 */
/* 52 */        OP_CALL, Ltak3-53,      /* ra rv */
/* 54 */        OP_RET, 0,              /* */
/* 56 *//* loop: */                     /* */
/* 56 */        OP_PUSH, NITER,         /* i */
/* 58 */        OP_CALL, 46-59,         /* i rv */
/* 60 */        OP_POP,                 /* i */
/* 61 */        OP_SUB1,                /* i' */
/* 62 */        OP_PUSH, 0,             /* i' 0 */
/* 64 */        OP_GET, 1,              /* i' 0 i' */
/* 66 */        OP_JGT, 58-67,          /* i' */
/* 68 */        OP_POP,                 /* */
/* 69 */        OP_HALT,                /* */
\end{verbatim}
\caption{The TAK program}
\label{fig:the-tak-program}
\end{figure}
%
\begin{figure}[htb]
\begin{verbatim}
op_get:
        x = sp[*pc++]; *--sp = x; L1: goto *pc++;
...
tak3:   ...
        &op_get, 2, /* BTB miss + update for goto in op_get */
        &op_get, 2, /* BTB miss + update for goto in op_get */
        &op_jgt, 5
...
\end{verbatim}
\caption{The problem with DTC}
\label{fig:the-problem-with-dtc}
\end{figure}
%
\begin{figure}[htb]
\begin{verbatim}
op_get(arg0):
        x = sp[arg0]; *--sp = x; L1: return;
...
tak3:   op_debug();
        op_get(2); /* RAS update here, RAS hit in op_get */
L2:     op_get(2); /* RAS update here, RAS hit in op_get */
L3:     op_jgt(5);
...
\end{verbatim}
\caption{The essence of CTC}
\label{fig:the-essence-of-ctc}
\end{figure}
%
\section{The Problem with Direct threading}
Consider the two GET instructions after the
\texttt{tak3} label at the start of the TAK
program in Figure~\ref{fig:the-tak-program}.
The direct threaded code for the GET instruction
and the two instruction instances is shown in
Figure~\ref{fig:the-problem-with-dtc}.
The sequence of events during the execution of
these two instructions is essentially the following:

\begin{enumerate}
\item load address from pc (address equals \texttt{op\_get})
\item increment pc
\item jump to the address (\texttt{op\_get})
\item execute \texttt{op\_get}
\item load address from pc (address equals \texttt{op\_get})
\item increment pc
\item L1: jump to the address (\texttt{op\_get});
the BTB now predicts \texttt{op\_get} as the target of the jump at L1
\item execute \texttt{op\_get}
\item load address from pc (address equals \texttt{op\_jgt})
\item increment pc
\item L1: jump to the address (\texttt{op\_jgt});
the BTB prediction for the jump at L1 (\texttt{op\_get}) differs from the
current dynamic target (\texttt{op\_jgt}), the processor suffers a branch
misprediction penalty, and the BTB is updated to predict \texttt{op\_jgt}
as the target for the jump at L1
\end{enumerate}
\par\noindent The fundamental problem is that the indirect jumps at
the end of the interpreter's handlers
will jump to different targets at different points in
time. This causes the BTB to frequently mispredict the
dynamic targets of these indirect jumps, which in turn
incurs significant performance penalties.
%
\section{The essence of call threading}
The call threaded code for the GET instruction
and the two instruction instances at the start
of the TAK program is shown in Figure~\ref{fig:the-essence-of-ctc}.
Th sequence of events during the execution of
these two instructions is essentially the following:

\begin{enumerate}
\item call \texttt{op\_get};
the return address L2 is stored in the return address location
(in a register or on the stack), and is also pushed on the RAS
\item execute \texttt{op\_get}
\item return;
the predicted target (L2) is popped off the RAS and execution
continues there, the dynamic target (L2) is retrieved (from the
stack or a register), and no misprediction occurs
\item call \texttt{op\_get};
the return address L3 is stored in the return address location
(in a register or on the stack), and is also pushed on the RAS
\item execute \texttt{op\_get}
\item return;
the predicted target (L3) is popped off the RAS and execution
continues there, the dynamic target (L3) is retrieved (from the
stack or a register), and no misprediction occurs
\item call \texttt{op\_jgt}
\end{enumerate}
\par\noindent In essence, each dynamic invokation has its own
branch prediction entry thanks to the RAS, which allows different
instruction instances to continue at different targets without
any branch mispredictions.
%
\section{Implementing Call--threading}
%
\section{Adapting call--threading to one-big-function interpreters}
%
\section{Experimental evaluation}

\begin{figure}[htb]
\begin{center}
\begin{tabular}{|@{~}l|r|r|r|r|}
\hline
Version & Clocks & Insns & CPI & BMR \\
\hline
C    &  206 &  296 & 0.695 & 0.052 \\
BC   & 2913 & 1650 & 1.765 & 0.223 \\
DTC  & 2226 & 1330 & 1.673 & 0.329 \\
CTC0 & 1302 & 1176 & 1.107 & 0.041 \\
CTC1 & 1236 & 1071 & 1.154 & 0.041 \\
CTC2 &  920 &  965 & 0.953 & 0.041 \\
CTC3 & 1837 & 1427 & 1.287 & 0.082 \\
CTC4 & 1562 & 1306 & 1.196 & 0.041 \\
CTC5 & 1060 & 1200 & 0.883 & 0.041 \\
CTC6 &  954 &  941 & 1.014 & 0.041 \\
\hline
\end{tabular}
\end{center}
\caption{K8 in 32-bit mode measurements}
\label{fig:measurements-k8-32}
\end{figure}

\begin{figure}[htb]
\begin{center}
\begin{tabular}{|@{~}l|r|r|r|r|}
\hline
Version & Clocks & Insns & CPI & BMR \\
\hline
C    &  201 &  296 & 0.680 & 0.052 \\
BC   & 3262 & 1906 & 1.712 & 0.223 \\
DTC  & 2399 & 1427 & 1.680 & 0.252 \\
CTC0 &  982 & 1127 & 0.871 & 0.040 \\
CTC1 &  921 & 1034 & 0.891 & 0.041 \\
CTC2 &  921 & 1034 & 0.891 & 0.041 \\
CTC3 & 1128 & 1387 & 0.814 & 0.040 \\
CTC4 & 1071 & 1310 & 0.818 & 0.041 \\
CTC5 & 1071 & 1310 & 0.818 & 0.041 \\
CTC6 & 1035 & 1050 & 0.985 & 0.041 \\
\hline
\end{tabular}
\end{center}
\caption{K8 in 64-bit mode measurements}
\label{fig:measurements-k8-64}
\end{figure}

\begin{figure}[htb]
\begin{center}
\begin{tabular}{|@{~}l|r|r|r|r|}
\hline
Version & Clocks & Insns & CPI & BMR \\
\hline
C    &  306 &  296 & 1.035 & 0.076 \\
BC   & 3256 & 1650 & 1.973 & 0.197 \\
DTC  & 2979 & 1330 & 2.240 & 0.282 \\
CTC0 & 2076 & 1176 & 1.766 & 0.051 \\
CTC1 & 1819 & 1071 & 1.699 & 0.039 \\
CTC2 & 1794 &  965 & 1.859 & 0.039 \\
CTC3 & 2128 & 1427 & 1.491 & 0.051 \\
CTC4 & 1819 & 1306 & 1.393 & 0.039 \\
CTC5 & 1844 & 1200 & 1.536 & 0.039 \\
CTC6 & 1772 &  941 & 1.884 & 0.039 \\
\hline
\end{tabular}
\end{center}
\caption{P4 measurements}
\label{fig:measurements-p4}
\end{figure}

\begin{figure}[htb]
\begin{center}
\begin{tabular}{|@{~}l|r|r|r|r|}
\hline
Version & Clocks & Insns & CPI & BMR \\
\hline
C    &  259 &  296 & 0.876 & 0.074 \\
BC   & 2356 & 1650 & 1.428 & 0.225 \\
DTC  & 2127 & 1330 & 1.599 & 0.332 \\
CTC0 & 1420 & 1176 & 1.207 & 0.061 \\
CTC1 & 1247 & 1071 & 1.165 & 0.044 \\
CTC2 & 1048 &  965 & 1.086 & 0.044 \\
CTC3 & 1447 & 1427 & 1.014 & 0.059 \\
CTC4 & 1286 & 1306 & 0.985 & 0.044 \\
CTC5 & 1184 & 1200 & 0.986 & 0.044 \\
CTC6 & 1025 &  941 & 1.089 & 0.044 \\
\hline
\end{tabular}
\end{center}
\caption{P6 Model 11 measurements}
\label{fig:measurements-p6m11}
\end{figure}

\begin{figure}[htb]
\begin{center}
\begin{tabular}{|@{~}l|r|r|r|r|}
\hline
Version & Clocks & Insns & CPI & BMR \\
\hline
C    &  259 &  296 & 0.876 & 0.074 \\
BC   & 2356 & 1650 & 1.428 & 0.225 \\
DTC  & 2126 & 1330 & 1.598 & 0.332 \\
CTC0 & 1426 & 1176 & 1.213 & 0.061 \\
CTC1 & 1251 & 1071 & 1.168 & 0.044 \\
CTC2 & 1047 &  965 & 1.085 & 0.044 \\
CTC3 & 1448 & 1427 & 1.015 & 0.059 \\
CTC4 & 1228 & 1306 & 0.987 & 0.044 \\
CTC5 & 1185 & 1200 & 0.987 & 0.044 \\
CTC6 & 1026 &  941 & 1.090 & 0.044 \\
\hline
\end{tabular}
\end{center}
\caption{P6 Model 8 measurements}
\label{fig:measurements-p6m8}
\end{figure}

\begin{figure}[htb]
\begin{center}
\begin{tabular}{|@{~}l|r|r|r|r|}
\hline
Version & Clocks & Insns & CPI & BMR \\
\hline
C    &  259 &  296 & 0.876 & 0.074 \\
BC   & 2357 & 1650 & 1.428 & 0.225 \\
DTC  & 2109 & 1330 & 1.586 & 0.332 \\
CTC0 & 1426 & 1176 & 1.213 & 0.061 \\
CTC1 & 1251 & 1071 & 1.168 & 0.044 \\
CTC2 & 1047 &  965 & 1.085 & 0.044 \\
CTC3 & 1448 & 1427 & 1.015 & 0.059 \\
CTC4 & 1288 & 1306 & 0.987 & 0.044 \\
CTC5 & 1185 & 1200 & 0.987 & 0.044 \\
CTC6 & 1026 &  941 & 1.090 & 0.044 \\
\hline
\end{tabular}
\end{center}
\caption{P6 Model 7 measurements}
\label{fig:measurements-p6m7}
\end{figure}

\begin{figure}[htb]
\begin{center}
\begin{tabular}{|@{~}l|r|r|r|r|}
\hline
Version & Clocks & Insns & CPI & BMR \\
\hline
C    &  259 &  296 & 0.876 & 0.074 \\
BC   & 2357 & 1650 & 1.428 & 0.225 \\
DTC  & 2126 & 1330 & 1.598 & 0.332 \\
CTC0 & 1426 & 1176 & 1.213 & 0.061 \\
CTC1 & 1251 & 1071 & 1.169 & 0.044 \\
CTC2 & 1047 &  965 & 1.085 & 0.044 \\
CTC3 & 1448 & 1427 & 1.015 & 0.059 \\
CTC4 & 1288 & 1306 & 0.987 & 0.044 \\
CTC5 & 1185 & 1200 & 0.987 & 0.044 \\
CTC6 & 1026 &  941 & 1.090 & 0.044 \\
\hline
\end{tabular}
\end{center}
\caption{P6 Model 3 measurements}
\label{fig:measurements-p6m3}
\end{figure}

\begin{figure}[htb]
\begin{center}
\begin{tabular}{|@{~}l|r|r|r|r|}
\hline
Version & Clocks & Insns & CPI \\
\hline
C    &  395 &  368 & 1.073 \\
BC   & 2952 & 2138 & 1.381 \\
DTC  & 2182 & 1187 & 1.837 \\
CTC0 & 1261 &  925 & 1.363 \\
CTC1 & 1072 &  808 & 1.328 \\
CTC2 & 1072 &  808 & 1.328 \\
CTC3 & 1812 & 1222 & 1.482 \\
CTC4 & 1578 & 1087 & 1.452 \\
CTC5 & 1578 & 1087 & 1.452 \\
CTC6 & 1099 &  808 & 1.361 \\
\hline
\end{tabular}
\end{center}
\caption{MPC7455 measurements}
\label{fig:measurements-mpc7455}
\end{figure}

\begin{figure}[htb]
\begin{center}
\begin{tabular}{|@{~}l|r|r|r|r|}
\hline
Version & Clocks & Insns & CPI \\
\hline
C    &  235 &  199 & 1.184 \\
BC   & 3479 & 2405 & 1.447 \\
DTC  & 2654 & 1481 & 1.788 \\
CTC0 & 1571 & 1275 & 1.233 \\
CTC1 & 1373 & 1167 & 1.176 \\
CTC2 & 1373 & 1167 & 1.176 \\
CTC3 & 1577 & 1407 & 1.121 \\
CTC4 & 1331 & 1318 & 1.009 \\
CTC5 & 1331 & 1318 & 1.009 \\
CTC6 & 1331 & 1318 & 1.009 \\
\hline
\end{tabular}
\end{center}
\caption{UltraSPARC IIIi measurements}
\label{fig:measurements-us3i}
\end{figure}

\begin{figure}[htb]
\begin{center}
\begin{tabular}{|@{~}l|r|r|r|r|}
\hline
Version & Clocks & Insns & CPI \\
\hline
C    &  251 &  304 & 0.826 \\
BC   & 3075 & 1930 & 1.593 \\
DTC  & 2215 & 1436 & 1.542 \\
CTC0 & 1473 & 1200 & 1.228 \\
CTC1 & 1279 & 1095 & 1.168 \\
CTC2 & 1007 &  989 & 1.018 \\
CTC3 & 1851 & 1452 & 1.275 \\
CTC4 & 1637 & 1330 & 1.231 \\
CTC5 & 1546 & 1225 & 1.262 \\
CTC6 & 1363 &  965 & 1.412 \\
\hline
\end{tabular}
\end{center}
\caption{K6-III measurements}
\label{fig:measurements-k63}
\end{figure}

\begin{figure}[htb]
\begin{center}
\begin{tabular}{|@{~}l|r|r|r|r|}
\hline
Version & Clocks & Insns & CPI & BMR \\
\hline
C    &  211 &  304 & 0.693 & 0.085 \\
BC   & 2089 & 1930 & 1.082 & 0.225 \\
DTC  & 1641 & 1436 & 1.143 & 0.264 \\
CTC0 & 1701 & 1200 & 1.417 & 0.113 \\
CTC1 & 1498 & 1095 & 1.368 & 0.056 \\
CTC2 & 1182 &  989 & 1.195 & 0.056 \\
CTC3 & 1476 & 1452 & 1.016 & 0.096 \\
CTC4 & 1372 & 1330 & 1.032 & 0.107 \\
CTC5 & 1343 & 1225 & 1.097 & 0.107 \\
CTC6 & 1204 &  965 & 1.248 & 0.095 \\
\hline
\end{tabular}
\end{center}
\caption{P5MMX measurements}
\label{fig:measurements-p5mmx}
\end{figure}

\begin{figure}[htb]
\begin{center}
\begin{tabular}{|@{~}l|r|r|r|r|}
\hline
Version & Clocks & Insns & CPI & BMR \\
\hline
C    &  247 &  304 & 0.811 & 0.322 \\
BC   & 1962 & 1930 & 1.016 & 0.238 \\
DTC  & 1540 & 1436 & 1.073 & 0.294 \\
CTC0 & 2308 & 1200 & 1.923 & 0.716 \\
CTC1 & 2220 & 1095 & 2.028 & 0.798 \\
CTC2 & 1846 &  989 & 1.866 & 0.798 \\
CTC3 & 2066 & 1452 & 1.423 & 0.696 \\
CTC4 & 1994 & 1330 & 1.499 & 0.798 \\
CTC5 & 1946 & 1225 & 1.589 & 0.798 \\
CTC6 & 1805 &  965 & 1.871 & 0.798 \\
\hline
\end{tabular}
\end{center}
\caption{P5 measurements}
\label{fig:measurements-p5}
\end{figure}

%
\section{Related work}
\cite{Bell73,Dewar75,Ritter80,Klint81,Kogge82,ErtlG01,ErtlG03:PLDI,ErtlG03:JILP,HoogerbruggeATW99,HoogerbruggeA00,DavisBCGW03,Proebsting95:POPL,BhandarkarD97,Costa99:ppdp,NassenCS01,PiumartaR98,IA32OPT,Rodriguez93:tcj59,Curley93,ShiGBE05,BerndlVZB05,RossiS96,VitaleA05,ErtlG04,SullivanBBGA03}
%
\section{Conclusions}
%
\bibliographystyle{plain}
\bibliography{mikpe}
%
\end{document}
